% Template: flow chart
\documentclass{article}

\usepackage{calc}
\usepackage{ifthen}
\usepackage{tikz}

\pagestyle{empty}
\usetikzlibrary{positioning, shapes.geometric, arrows, shapes.misc}

\begin{document}
%for start point or end point use
\tikzstyle{terminal} = [rounded rectangle, draw, fill = black!0, 
                        text centered, node distance=2.6cm, minimum height=4em, 
                        minimum width = 10em]  
%for desision use
\tikzstyle{decision} = [diamond, draw, fill = blue!20, text centered, 
                        node distance = 2.6cm, minimum height = 5em, 
                        minimum width = 5em,aspect = 2]   
%for process use 
\tikzstyle{process} = [rectangle, draw, fill=yellow!20, text centered, 
                       rounded corners, node distance=2.6cm, minimum height=4em, 
                       minimum width = 10em]  
\begin{tikzpicture}
\node (START) [terminal] {\color{red}{START}};
\node (PROCESS) [process, below=of START] {PROCESS};
\node (DECISION1) [decision,below=of PROCESS] {DECISION 1};
\node (DECISION2) [decision,left=of DECISION1] {DECISION 2};
\node (DECISION3) [decision,below=of DECISION1] {DECISION 3};
\node (END) [terminal,below=of DECISION3] {\color{red}{END}};

\draw [-latex, thick, black] (START) -- (PROCESS);
\draw [-latex, thick, black] (PROCESS) -- (DECISION1);
\draw [-latex, thick, black] (DECISION1) -- node [pos = 0.5, fill = white] {no} (DECISION2);
\draw [-latex, thick, black] (DECISION2) |- node [pos = 0.25, fill = white] {no} (PROCESS);
\draw [-latex, thick, black] (DECISION2) -- ++(-2, 0) |- node [pos = 0.25,fill = white] {yes}(START);
\draw [-latex, thick, black] (DECISION1) -- node [pos = 0.5, fill = white]{yes} (DECISION3);
\draw [-latex, thick, black] (DECISION3) -- node[pos = 0.5, fill = white]{yes} (END);
\draw [-latex, thick, black] (DECISION3) -| node[pos = 0.25, fill = white]{no} (DECISION2);

\end{tikzpicture}
\end{document}
